\documentclass[]{elsarticle} %review=doublespace preprint=single 5p=2 column
%%% Begin My package additions %%%%%%%%%%%%%%%%%%%
\usepackage[hyphens]{url}

  \journal{Saudi Journal of Biological Sciences} % Sets Journal name


\usepackage{lineno} % add
  \linenumbers % turns line numbering on
\providecommand{\tightlist}{%
  \setlength{\itemsep}{0pt}\setlength{\parskip}{0pt}}

\usepackage{graphicx}
\usepackage{booktabs} % book-quality tables
%%%%%%%%%%%%%%%% end my additions to header

\usepackage[T1]{fontenc}
\usepackage{lmodern}
\usepackage{amssymb,amsmath}
\usepackage{ifxetex,ifluatex}
\usepackage{fixltx2e} % provides \textsubscript
% use upquote if available, for straight quotes in verbatim environments
\IfFileExists{upquote.sty}{\usepackage{upquote}}{}
\ifnum 0\ifxetex 1\fi\ifluatex 1\fi=0 % if pdftex
  \usepackage[utf8]{inputenc}
\else % if luatex or xelatex
  \usepackage{fontspec}
  \ifxetex
    \usepackage{xltxtra,xunicode}
  \fi
  \defaultfontfeatures{Mapping=tex-text,Scale=MatchLowercase}
  \newcommand{\euro}{€}
\fi
% use microtype if available
\IfFileExists{microtype.sty}{\usepackage{microtype}}{}
\bibliographystyle{elsarticle-harv}
\ifxetex
  \usepackage[setpagesize=false, % page size defined by xetex
              unicode=false, % unicode breaks when used with xetex
              xetex]{hyperref}
\else
  \usepackage[unicode=true]{hyperref}
\fi
\hypersetup{breaklinks=true,
            bookmarks=true,
            pdfauthor={},
            pdftitle={Comparative study on Performance of Different Substrates on Oyster Mushroom ()},
            colorlinks=false,
            urlcolor=blue,
            linkcolor=magenta,
            pdfborder={0 0 0}}
\urlstyle{same}  % don't use monospace font for urls

\setcounter{secnumdepth}{5}
% Pandoc toggle for numbering sections (defaults to be off)
% Pandoc header
\usepackage{siunitx}
\usepackage{moreverb}
% \usepackage{fontenc}
\usepackage{fontawesome}
\usepackage{booktabs}
\usepackage{longtable}
\usepackage{array}
\usepackage{multirow}
\usepackage{wrapfig}
\usepackage{float}
\usepackage{colortbl}
\usepackage{pdflscape}
\usepackage{tabu}
\usepackage{threeparttable}
\usepackage{threeparttablex}
\usepackage[normalem]{ulem}
\usepackage{makecell}
\usepackage{xcolor}
\usepackage{tikz} % required for image opacity change
\usepackage[absolute,overlay]{textpos} % for text formatting
\usepackage{chemfig}

\newcommand\BibTeX{{\rmfamily B\kern-.05em \textsc{i\kern-.025em b}\kern-.08em
T\kern-.1667em\lower.7ex\hbox{E}\kern-.125emX}}
\sisetup{per-mode=symbol}

% this is to sort and place the reference in nice order
\DeclareRobustCommand{\firstsecond}[2]{#2}

% \bibliographystyle{plainnat}
\usepackage[numbers]{natbib}



\begin{document}
\begin{frontmatter}

  \title{Comparative study on Performance of Different Substrates on Oyster
Mushroom (\textit{Pleurotus ostreatus})}
    \author[Department of Horticulture and Plant Protection]{Samita Paudel}
  
  
    \author[Department of Plant Breeding]{Deependra Dhakal\corref{c1}}
   \ead{ddhakal.rookie@gmail.com} 
   \cortext[c1]{Corresponding Author}
    
  \begin{abstract}
  Substrate type is an important factor determining growth and yield of
  oyster mushroom. Five different substrates namely rice straw, maize
  hulls, banana leaves, fingermillet husk and mixture of rice straw \&
  black gram pod shell (1:1) were evaluated for the yield and related
  attributes of \textit{Pleurotus ostreatus}. Standard cultivation
  practice was followed with steam sterilization and spawning was done on
  575 g of substrate in individual poly-bag. The data of three flushes
  were recorded. Our results revealed that full spawn run completed
  earlier (18.57 days) in fingermillet husk as compared to any other
  tested substrates. The highest total quantity yield was obtained in
  fingermillet husk (\SI{1024.57}{\gram \per bag}) and rice straw
  (\SI{956.14}{\gram \per bag}) with corresponding biological efficiency
  178.19\% and 166.29\% respectively which were significantly higher than
  all other treatments. The cropping duration was significantly higher in
  maize hulls and banana leaves as compared to rest of three treatments
  viz. fingermillet husk, rice straw and mixture of rice straw and black
  gram pod shell (1:1). These three treatments were not statistically
  different for cropping duration with each other. Considering the
  biological efficiency and earliness of crop the performance of
  fingermillet husk, followed by rice straw was found to be better.
  \end{abstract}
  
 \end{frontmatter}

\hypertarget{keywords}{%
\section*{Keywords}\label{keywords}}
\addcontentsline{toc}{section}{Keywords}

Agricultural byproducts; Biological efficiency; Cropping duration;
\textit{Pleurotus ostreatus}

\clearpage

\hypertarget{introduction}{%
\section{Introduction}\label{introduction}}

Oyster mushroom (\textit{Pleurotus species}) is an edible, saprophytic
and lignocellulolytic type of fungus belonging to the class
Agaricomycetes, order Agaricales and family Pleurotaceae. It is the
second widely cultivated mushroom following the Agaricus bisporus in the
world (Sańchez, 2010). There are over 70 species of oyster mushroom been
discovered (Kong, 2004) and still there are lots to explore.
\textit{Pleurotus ostreatus} is the most popular species of oyster
mushroom found in Nepal. The Latin word `Pleurotus' means beside the ear
and `ostreatus' means oyster shaped (Cohen et al., 2002) and in Nepal it
is often called ``Kanya chayu'' due to the ear like appearance.
Pleurotus species can efficiently degrade agricultural byproducts and
can grow on wide range of agricultural wastes. A substrate is any
material that serves as a medium of growth for a living thing in which
enzymes can act upon and break it to release nutrients for the growing
organism. There are a range of wastes that can be used for oyster
mushroom cultivation, but it depends on the basis of availability of the
substrate and its cost.

The availability of good substrate is an important requirement for the
better growth and higher yield of mushroom. An ideal substrate should
contain adequate amount of nitrogen and carbohydrates for rapid mushroom
growth (Khare et al., 2010). Total Carbon (C), total Nitrogen (N),
Carbon/Nitrogen ratio (C/N) are important factors that determines the
mycelium colonized and development of fruiting bodies in oyster
mushroom. Hong et al. (1981) have shown that in both Agaricus bitorquis
and \textit{Pleurotus ostreatus} the yield of mycelium decreases under
lower or higher C/N ratio. Pleurotus fungi mobilizes the carbohydrate
composed in rice straw mainly through cellulose and hemicellulose
degradation (Fazaeli et al., 2006). Rice straw is a popular substrate
for Pleurotus cultivation in Asia, mainly favored for its composition of
slow degrading carbohydrates. Rice straw has the chemical composition of
(in percentage dry mass basis) 0.96\% N, 73.01\% NDF, 41.59\% ADF,
31.42\% Hemicellulose, 33.35\% Cellulose and 4.84\% Acid detergent
lignin when sampled over different stages of growth, amounting to a
generalized carbon(lignocellulosic)-nitrogen ratio of 72\% (Sarnklong et
al., 2010). Similarly corn cob contains 47\% cellulose, 25\% lignin,
total Carbon 47\%, Nitrogen 0.48\% and C/N ratio of 97:1 that can be
used as substrate for Pleurotus cultivation (Wha Choi, 2004). Amongst
various cereal straws, paddy straw was reported to be the best substrate
for the cultivation of oyster mushroom (Khanna and Garcha, 1982). The
leaves and pseudostems of banana contain high levels of lignocellulose
(Reddy, 2001). These lignocellulose materials are efficient substrates
for white-rot fungi, which produce lignolytic and cellulolytic enzymes
(Pointing, 2001). Legume straws are rich in Nitrogen content and are
suitable as Pleurotus substrates (Poppe, 1995). However, selection of
right substrate to achieve high yield of oyster mushroom can be a
challenge as there seems to be a wide range of agricultural crop
residues available in which the oyster mushroom can be grown. The
objective of this study was therefore to determine the effectiveness of
different agricultural wastes (i.e.~rice straw, maize hulls, banana
leaves, finger millet husk and black gram pod shell) on the yield
performance and biological efficiency of \textit{Pleurotus ostreatus} in
the subtropical condition of midhills of Nepal.

\hypertarget{materials-and-method}{%
\section{Materials and Method}\label{materials-and-method}}

The research was conducted under a condition in Institute of Agriculture
and Animal Sciences, Lamjung, from December-2016 to March-2017. The
experiment was designed in a Completely Randomized Design (CRD) with
five treatments (i.e.~Rice straw, maize hulls, banana leaves,
fingermillet husk, mixture of rice straw and black gram pod shell (1:1))
and seven replications per treatment.

The substrates were chopped to about 3-5 cm in length and soaked
overnight in the tank filled with water. Steam sterilization method was
followed for at least 15-20 minutes in a Metallic drum to reach
temperature up to 90 \(^\circ C\). Then, the substrates were spread over
sterilized clean plastic sheet for air cooling below 25 \(^\circ C\).
Transparent poly-bags were taken for filling of substrate in clean and
sterile condition. The moisture content of the substrates while filling
was around 60\%. The substrates were filled on dry weight basis i.e.575
g per poly-bag. The spawning was done at the rate of 10\% on dry weight
basis. Three layer of spawning was done, one at bottom, another at the
mid-section and lastly at top, starting from the bottom layer. After
spawning the bag was tightly knocked with rope. The spawn in the
substrate should be clearly seen from outside. The bags after spawning
were weight and their respective weights maintained. The bags were then
perforated in 8-10 numbers with sterilized needle to permit air
circulation.

The bags were moved to a production room after spawning and were hanged
randomly. The room was made completely dark using black poly ethylene
sheets. The temperature was around 17-20 \(^\circ C\) and relative
humidity around 90\%. For the first 15 days of spawn run the artificial
lighting was not provided. After the proper development of white
mycelium, the polythene covers were removed. At the end of the spawn
run, for pinning dim light along with sufficient fresh air was
introduced in to the room through ventilation and \(\mathrm{CO_2}\)
concentration was lowered. The temperature and relative humidity was
maintained by spraying water twice a day on the ball of mushroom and on
the floor of the room. The insecticide Nuvan was sprayed in substrate to
avoid the appearance of several insects. The application of Nuvan was
done only before pin head appearance or after harvesting of crop. The
harvesting of mushroom was done when the cap began to fold. The picking
was done by twisting the mushroom gently and pulling out, leaving any
stub. Cropping was done up to three flushes.

The Parameters taken were days taken for full spawn run (days), Quantity
harvest in three flushes (g), Days taken for first and second harvest,
cropping duration (days) and total quantity harvest (g). The fruiting
bodies after harvested are weighted and measurement of mushroom was
taken. The biological efficiency of mushroom per gram of substrate on
dry wt. basis was calculated by using the following formula:

\[
\textrm{B.E.} \% =\frac{\text{TQH from each bag}}{\textrm{DW of substrate on each bag}} \times 100
\]

Where,

TQH/bag: Total quantity harvest from each bag

DW: dry weight of substrate on each bag

The difference of mean was compared, using Tukey's test at the level of
5\% significance.

\hypertarget{results-and-discussion}{%
\section{Results and Discussion}\label{results-and-discussion}}

The results obtained from the studies on use of different substrates for
the cultivation of \textit{Pleurotus ostreatus} in the growth
performance and yield is presented in the tables and discussed below.

\hypertarget{days-taken-for-full-spawn-run}{%
\subsection{Days taken for full spawn
run}\label{days-taken-for-full-spawn-run}}

The number of days taken for full spawn run ranged from 18.57 days to
23.29 days on different substrates (Table \ref{tab:substrate-days}).
Significantly lowest number of days for full spawn run was recorded on
fingermillet husk (18.57 days) and the mixture of rice straw and black
gram pod shell (1:1) (19.0 days). Banana leaves (23.29 days) and rice
straw (22.14 days) took the highest number of days for completion of
full spawn run followed by maize hulls (20.86 days). Our results were
almost similar to the findings of Shah et al. (2004) who reported that
the spawn running took in 16-25 days after inoculation. This variation
in number of days taken for full spawn run in different substrates could
be due to the variations in chemical composition and Carbon to Nitrogen
ratio (C: N) of the substrates used (Bhatti et al., 2007).

\hypertarget{days-taken-for-first-and-second-harvest}{%
\subsection{Days taken for first and second
harvest}\label{days-taken-for-first-and-second-harvest}}

The days taken for first harvest of mushroom (Table
\ref{tab:substrate-days}) were significantly affected by substrates
used. The average numbers of days taken for first harvest were between
38-46 days. The earliest harvest was noted on fingermillet husk (38d
ays) and mixture of rice straw: black gram pod (1:1) (38.71 days)
followed by maize hulls (40.14 days) and rice straw (40.86 days). Banana
leaves took significantly the highest number of days (46.86 days) for
first harvest. Contrary to our study Quimio et al. (1990) reported that
good harvest of \textit{P. ostreatus} was obtained 3-4 weeks after spawn
inoculation.

Similarly, the average numbers of days taken for second harvest were
between 59-60 days. Fingermillet husk (51.29 days), rice straw (53.14
days) and mixture of rice straw and black gram pod shell (53.14 days)
took minimum days for second harvest. Whereas banana leaves (69.29 days)
and maize hulls (67.86 days) took the maximum number of days to give
second harvest. The early harvest of mushroom in fingermillet husk, rice
straw might be due to the availability of nutrients required for the
mushroom growth particularly for its spawn run and pin head development
was supplied by substrates which decomposed quicker compared to other
tested substrates.

\hypertarget{cropping-duration}{%
\subsection{Cropping duration}\label{cropping-duration}}

The data recorded for cropping duration of oyster mushroom
\ref{tab:substrate-days} indicates highly significant difference between
tested substrates. The cropping duration was significantly longest on
maize hulls (96.86 days) and banana leaves (94.43 days). Whereas the
shortest cropping duration was recorded in fingermillet husk (69.57
days), rice straw (72.43 days) and mixture of rice straw and black gram
pod shell (1:1) (71 days) which were not significantly different with
each other. The cropping duration in our study ranged from 69 days to 96
days which is similar to findings of Khanna and Garcha (1982) that it
may take up-to 104 days to harvest yield from oyster mushroom grown on
paddy straw. The variation in cropping period among different substrates
could emanate from variations in the time elapsed in formation of
pinheads, maturation of fruiting bodies, interval between flushes,
number of flushes and yield, which in turn is affected by the nature of
the substrates (Chang et al., 1981).

\begin{table}[t]

\caption{\label{tab:substrate-days}Effect of different substrates on days taken for full spawn run, days taken for first and second harvest and cropping duration of \textit{Pleurotus ostreatus}.}
\centering
\fontsize{8}{10}\selectfont
\begin{tabular}{>{\raggedright\arraybackslash}p{7em}>{\raggedright\arraybackslash}p{7em}>{\raggedright\arraybackslash}p{7em}>{\raggedright\arraybackslash}p{7em}>{\raggedright\arraybackslash}p{7em}}
\toprule
Substrates & Days taken for full spawn run & Days taken for first harvest & Days taken for second harvest & Cropping duration\\
\midrule
Rice straw & 22.14cd & 40.86c & 53.14a & 72.43a\\
Maize hulls & 20.86bc & 40.14bc & 67.86b & 96.86b\\
Banana leaves & 23.29d & 46.86d & 69.29b & 94.43b\\
Rice straw: black gram pod shell (1:1) & 19.00ab & 38.71ab & 53.14a & 71.00a\\
Fingermillet husk & 18.57a & 38.00a & 51.29a & 69.57a\\
\addlinespace
Significance at 5\% & ** & ** & * & *\\
\bottomrule
\multicolumn{5}{l}{\textsuperscript{a} The symbols * denotes significant and ** denotes highly significant}\\
\end{tabular}
\end{table}

\hypertarget{quantity-harvest-three-flushes-and-total-quantity-harvest}{%
\subsection{Quantity harvest (three flushes) and Total quantity
harvest}\label{quantity-harvest-three-flushes-and-total-quantity-harvest}}

The quantity harvest varied significantly on the different substrates in
all of the three flushes (Table 2). In the first flush, significantly
highest quantity harvest (571.43 g) was obtained on fingermillet husk
followed by rice straw (453.57 g) and mixture of rice straw and black
gram pod shell (1:1) (426.86 g). The lowest yield was obtained on banana
leaves (256.71 g) and maize hulls (258.86 g).

In the second flush, rice straw (369 g) and fingermillet husk (328.29 g)
produced the highest yield followed by the mixture of rice straw and
black gram pod shell (1:1) (241.43 g). In the same flush, the lowest
yield was obtained on maize hulls (151.43 g) and banana leaves (218.71
g).

In the third flush highest yield was obtained on rice straw (133.57 g),
mixture of rice straw and black gram pod shell (1:1) (127.71 g) and
fingermillet husk (124.86 g) respectively. In the same flush, the lowest
yield was recorded on maize hulls (45.27 g) and banana leaves (53.43 g).

The total quantity harvest result showed the significant difference
between the tested substrates \ref{tab:substrate-quantity}. Fingermillet
husk (1024.57 g) and rice straw (956.14 g) gave significantly higher
total quantity harvest followed by mixture of rice straw and black gram
pod shell (1:1) (796 g). Maize hulls and banana leaves gave the lowest
total quantity harvest with corresponding yield of 455.56 g and 528.86 g
respectively and had no significant difference with each other.
Fingermillet husk, rice straw followed by the mixture of rice straw and
black gram pod shell gave higher quantity harvest in all of the three
flushes and resulted in higher total quantity harvest and were
considered as best substrates. Similar results were obtained by (Khanna
and Garcha, 1982) who reported paddy straw as the best substrate for the
cultivation of oyster mushroom amongst various cereal straws. The higher
yield of mushroom in rice straw is due to easier way of getting
nutrients from the cellulosic substances (Ponmurugan et al., 2007). The
higher yield of oyster mushroom in fingermillet husk and mixture of rice
straw and black gram pod shell (1:1) could be due to the better
availability of nutrients from them. Whereas the poor yield under maize
hulls and banana leaves might be due to their lower availability of
nutrients and low water holding capacity which is due to higher lignin
content on them.

\hypertarget{biological-efficiency}{%
\subsection{Biological efficiency}\label{biological-efficiency}}

Biological efficiency is used to evaluate the efficiency of conversion
of substrate in to mushrooms. It was determined as ratio of the Total
Quantity Harvest to the dry weight of each substrate. In our experiment
we took 0.575 kg of dry substrates for each bag. Results of the
biological efficiency varied significantly among the substrates used
\ref{tab:substrate-quantity}. The significantly highest percentage of
biological efficiency was observed on fingermillet husk and rice straw
with corresponding biological efficiency of 178.19\% and 166.29\%
respectively followed by the mixture of rice straw and black gram pod
(1:1) (138.43\%). Whereas the least biological efficiency of 79.23\% and
91.98\% was observed in maize hulls and banana leaves respectively and
they were not significantly different with each other. Núñez and Mendoza
(2002) reported that the biological efficiency of the studied substrates
varied from 50.8\% to 106.2\% while working with \textit{P. ostreatus}
however our results are not consistent with theirs. In our study we got
biological efficiency greater than 100\% which is supported by Contreras
et al. (2004) and Mandeel et al. (2005) who reported that in the case of
\textit{P. ostreatus}, it is possible to reach a biological efficiency
greater than 100\%. Similarly, Jiskani (1999) also reported that one kg
of dry substrate can produce one kg of fresh mushroom which is the 100\%
dry weight of substrate used. However, these values can vary
substantially, depending on the type of substrate and cultivation
strategy. The variation in biological efficiency of oyster mushroom is
mostly due to the Physiochemical properties of substrate used.
Fingermillet husk and rice straw contain adequate C/N ratio for the
better production of \textit{Pleurotus ostreatus} compared to the rest
substrates. Since, the yield of \textit{Pleurotus ostreatus} mycelium
decreases under lower or higher C/N ratio (Warcup, 1951).

\begin{table}[t]

\caption{\label{tab:substrate-quantity}Effect of different substrates on quantity harvests of \textit{Pleurotus ostreatus} during the three flushes}
\centering
\fontsize{8}{10}\selectfont
\begin{tabular}{>{\raggedright\arraybackslash}p{8em}>{\raggedright\arraybackslash}p{6em}>{\raggedright\arraybackslash}p{6em}>{\raggedright\arraybackslash}p{6em}>{\raggedright\arraybackslash}p{6em}>{\raggedright\arraybackslash}p{6em}}
\toprule
Substrates & First Quantity harvest 
QH1 & Second Quantity harvest
QH2 & Third Quantity harvest
QH3 & Total Quantity Harvest (gm/bag) & Biological efficiency (\%)\\
\midrule
Rice straw & 453.57b & 369.00a & 133.57a & 956.14a & 166\\
Maize hulls & 258.86c & 151.43c & 45.27b & 455.56c & 79\\
Banana Leaves & 256.71c & 218.71c & 53.43b & 528.86c & 92\\
Mixture of rice straw and black gram pod shell (1:1) & 426.86b & 241.43bc & 127.71a & 796.00b & 138\\
Fingermillet husk & 571.43a & 328.29ab & 124.86a & 1024.57a & 178\\
\addlinespace
Significance at 5\% & ** & ** & * & ** & \\
\bottomrule
\multicolumn{6}{l}{\textsuperscript{a} The symbols * denotes significant and ** denotes highly significant}\\
\end{tabular}
\end{table}

\hypertarget{conclusion}{%
\section{Conclusion}\label{conclusion}}

From the present study it is confirmed that oyster mushroom
(\textit{Pleurotus ostreatus}) can be cultivated on rice straw, maize
hulls, banana leaves, fingermillet husk and mixture of rice straw \&
black gram pod (1:1) with varying growth performances. Fingermillet husk
and rice straw were identified as the most suitable substrates for
oyster mushroom cultivation. Fingermillet husk and rice straw followed
by mixture of rice straw and black gram pod shell (1:1) produced a
significantly higher yield and biological efficiency in shorter cropping
duration. Fingermillet husk and mixture of rice straw and black gram pod
shell (1:1) also proved to be better in terms of days taken for full
spawn run. Therefore, fingermillet husk and rice straw can be
recommended as the preferred substrate for oyster mushroom cultivation.
In addition, mixture of rice straw \& black gram pod (1:1) can be used
as alternative substrate given that the growth performance, yield and
cropping duration of oyster mushroom was better in it next to
fingermillet husk and rice straw. The utilization of fingermillet husk
and rice straw as substrates for oyster mushroom cultivation can be a
solution to the huge agricultural by products available. And yet,
further studies need to be conducted on the potentials of various
agricultural wastes on oyster mushroom cultivation.

\hypertarget{references}{%
\section*{References}\label{references}}
\addcontentsline{toc}{section}{References}

\hypertarget{refs}{}
\leavevmode\hypertarget{ref-bhatti2007growth}{}%
Bhatti, M., Jiskani, M., Wagan, K., Pathan, M., Magsi, M., 2007. Growth,
development and yield of oyster mushroom, pleurotus ostreatus (jacq. Ex.
Fr.) kummer as affected by different spawn rates. Pak. J. Bot 39,
2685--2692.

\leavevmode\hypertarget{ref-chang1981cultivation}{}%
Chang, S., Lau, O., Cho, K., 1981. The cultivation and nutritional value
of pleurotus sajor-caju. European journal of applied microbiology and
biotechnology 12, 58--62.

\leavevmode\hypertarget{ref-cohen2002}{}%
Cohen, R., Persky, L., Hadar, Y., 2002. Biotechnological applications
and potential of wood-degrading mushrooms of the genus pleurotus.
Applied microbiology and biotechnology 58, 582--594.

\leavevmode\hypertarget{ref-contreras2004soaking}{}%
Contreras, E., Sokolov, M., Mejı'a, G., Sánchez, J., 2004. Soaking of
substrate in alkaline water as a pretreatment for the cultivation of
pleurotus ostreatus. The Journal of Horticultural Science and
Biotechnology 79, 234--240.

\leavevmode\hypertarget{ref-fazaeli2006nutritive}{}%
Fazaeli, H., Azizi, A., Amile, M., 2006. Nutritive value index of
treated wheat straw with pleurotus fungi fed to sheep. Pak. J. Biol. Sci
9, 2444--2449.

\leavevmode\hypertarget{ref-hong1981}{}%
Hong, J.-S., Lee, K.-S., Choi, D.-S., 1981. Studies on basidiomycetes
(i)-on the mycelial growth of agaricus bitorquis and pleurotus
ostreatus. The Korean Journal of Mycology 9, 19--24.

\leavevmode\hypertarget{ref-jiskani1999brief}{}%
Jiskani, M., 1999. A brief outline t'he fungi('cultivation of
mushrooms). Izhar Pub. Tandojam, Pakistan 94.

\leavevmode\hypertarget{ref-khanna1982utilization}{}%
Khanna, P., Garcha, H., 1982. Utilization of paddy straw for cultivation
of pleurotus species. Mushroom Newsl. Trop 2, 5--9.

\leavevmode\hypertarget{ref-khare2010}{}%
Khare, K., Mutuku, J., Achwania, O., Otaye, D., 2010. Production of two
oyster mushrooms, pleurotus sajor-caju and p. Florida on supplemented
and un-supplemented substrates. International Journal of Agriculture and
Applied Sciences 6, 4--11.

\leavevmode\hypertarget{ref-kong2004}{}%
Kong, W.-S., 2004. Descriptions of commercially important pleurotus
species. Oyster mushroom cultivation. Part II. Oyster mushrooms. Seoul:
Heineart Incorporation 54--61.

\leavevmode\hypertarget{ref-mandeel2005cultivation}{}%
Mandeel, Q., Al-Laith, A., Mohamed, S., 2005. Cultivation of oyster
mushrooms (pleurotus spp.) on various lignocellulosic wastes. World
Journal of Microbiology and Biotechnology 21, 601--607.

\leavevmode\hypertarget{ref-nunez2002submerged}{}%
Núñez, J.P., Mendoza, C.G., 2002. Submerged fermentation of
lignocellulosic wastes under moderate temperature conditions for oyster
mushroom growing substrates. Mushroom Biology and Mushroom Products 5,
545--549.

\leavevmode\hypertarget{ref-pointing2001}{}%
Pointing, S., 2001. Feasibility of bioremediation by white-rot fungi.
Applied microbiology and biotechnology 57, 20--33.

\leavevmode\hypertarget{ref-ponmurugan2007effect}{}%
Ponmurugan, P., Sekhar, Y.N., Sreesakthi, T., 2007. Effect of various
substrates on the growth and quality of mushrooms. Pakistan Journal of
Biological Sciences 10, 171--173.

\leavevmode\hypertarget{ref-poppe1995}{}%
Poppe, J., 1995. Cultivation of edible mushrooms on tropical
agricultural wastes. University of Gent, Gent.

\leavevmode\hypertarget{ref-quimio1990technical}{}%
Quimio, T., Chang, S.-t., Royse, D.J., others, 1990. Technical
guidelines for mushroom growing in the tropics.

\leavevmode\hypertarget{ref-reddy2001bioconversion}{}%
Reddy, G.V., 2001. Bioconversion of banana waste into protein by two
pleurotus species p ostreatus and p sajor caju biotechnological approach
(Thesis). Sardar Patel University, Department of Bio Science.

\leavevmode\hypertarget{ref-sanchez2010cultivation}{}%
Sańchez, C., 2010. Cultivation of pleurotus ostreatus and other edible
mushrooms. Applied microbiology and biotechnology 85, 1321--1337.

\leavevmode\hypertarget{ref-sarnklong2010utilization}{}%
Sarnklong, C., Cone, J., Pellikaan, W., Hendriks, W., 2010. Utilization
of rice straw and different treatments to improve its feed value for
ruminants: A review. Asian-Australasian Journal of Animal Sciences 23,
680--692.

\leavevmode\hypertarget{ref-shah2004comparative}{}%
Shah, Z., Ashraf, M., Ishtiaq, M., 2004. Comparative study on
cultivation and yield performance of oyster mushroom (pleurotus
ostreatus) on different substrates (wheat straw, leaves, saw dust).
Pakistan Journal of Nutrition 3, 158--160.

\leavevmode\hypertarget{ref-warcup1951studies}{}%
Warcup, J., 1951. Studies on the growth of basidiomycetes in soil.
Annals of Botany 15, 305--318.

\leavevmode\hypertarget{ref-kyung2004shelfcultivation}{}%
Wha Choi, K., 2004. Shelf cultivation of oyster mushroom with emphasis
on substrate fermentation. Mushroom GrowersH'andbook 1: Oyster Mushroom
Cultivation MushWorld.


\end{document}


